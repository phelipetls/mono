\documentclass[]{article}
\usepackage{lmodern}
\usepackage{amssymb,amsmath}
\usepackage{ifxetex,ifluatex}
\usepackage{fixltx2e} % provides \textsubscript
\ifnum 0\ifxetex 1\fi\ifluatex 1\fi=0 % if pdftex
  \usepackage[T1]{fontenc}
  \usepackage[utf8]{inputenc}
\else % if luatex or xelatex
  \ifxetex
    \usepackage{mathspec}
  \else
    \usepackage{fontspec}
  \fi
  \defaultfontfeatures{Ligatures=TeX,Scale=MatchLowercase}
\fi
% use upquote if available, for straight quotes in verbatim environments
\IfFileExists{upquote.sty}{\usepackage{upquote}}{}
% use microtype if available
\IfFileExists{microtype.sty}{%
\usepackage{microtype}
\UseMicrotypeSet[protrusion]{basicmath} % disable protrusion for tt fonts
}{}
\usepackage[margin=1in]{geometry}
\usepackage{hyperref}
\hypersetup{unicode=true,
            pdftitle={Processo de testes econométricos},
            pdfborder={0 0 0},
            breaklinks=true}
\urlstyle{same}  % don't use monospace font for urls
\usepackage{graphicx,grffile}
\makeatletter
\def\maxwidth{\ifdim\Gin@nat@width>\linewidth\linewidth\else\Gin@nat@width\fi}
\def\maxheight{\ifdim\Gin@nat@height>\textheight\textheight\else\Gin@nat@height\fi}
\makeatother
% Scale images if necessary, so that they will not overflow the page
% margins by default, and it is still possible to overwrite the defaults
% using explicit options in \includegraphics[width, height, ...]{}
\setkeys{Gin}{width=\maxwidth,height=\maxheight,keepaspectratio}
\IfFileExists{parskip.sty}{%
\usepackage{parskip}
}{% else
\setlength{\parindent}{0pt}
\setlength{\parskip}{6pt plus 2pt minus 1pt}
}
\setlength{\emergencystretch}{3em}  % prevent overfull lines
\providecommand{\tightlist}{%
  \setlength{\itemsep}{0pt}\setlength{\parskip}{0pt}}
\setcounter{secnumdepth}{0}
% Redefines (sub)paragraphs to behave more like sections
\ifx\paragraph\undefined\else
\let\oldparagraph\paragraph
\renewcommand{\paragraph}[1]{\oldparagraph{#1}\mbox{}}
\fi
\ifx\subparagraph\undefined\else
\let\oldsubparagraph\subparagraph
\renewcommand{\subparagraph}[1]{\oldsubparagraph{#1}\mbox{}}
\fi

%%% Use protect on footnotes to avoid problems with footnotes in titles
\let\rmarkdownfootnote\footnote%
\def\footnote{\protect\rmarkdownfootnote}

%%% Change title format to be more compact
\usepackage{titling}

% Create subtitle command for use in maketitle
\providecommand{\subtitle}[1]{
  \posttitle{
    \begin{center}\large#1\end{center}
    }
}

\setlength{\droptitle}{-2em}

  \title{Processo de testes econométricos}
    \pretitle{\vspace{\droptitle}\centering\huge}
  \posttitle{\par}
    \author{}
    \preauthor{}\postauthor{}
    \date{}
    \predate{}\postdate{}
  

\begin{document}
\maketitle

\begin{center}\LARGE{Testes de raiz unitária}\end{center}

\includegraphics{etapa_de_testes_files/figure-latex/unnamed-chunk-2-1.pdf}

\newpage

\begin{center}\LARGE{Séries diagnosticadas como estacionárias em cada teste}\end{center}

\includegraphics{etapa_de_testes_files/figure-latex/unnamed-chunk-3-1.pdf}

\newpage

\begin{center}\LARGE{Testes de co-integração para várias combinações das variáveis}\end{center}\begin{center}\large{Philips-Ouliaris}\end{center}

\[ spread = \beta_0 + \beta_1selic + \beta_2inad + \beta_3pib\_mensal + \beta_4prod\_ind + \beta_5igp\_di \]

\begin{verbatim}
# A tibble: 2 x 3
  V1       V2         resultado
  <chr>    <chr>      <chr>    
1 ihh      pib_mensal 0.01     
2 prod_ind igp_di     0.01     
\end{verbatim}

\begin{verbatim}
# A tibble: 5 x 4
  V1    V2         V3         resultado
  <chr> <chr>      <chr>      <chr>    
1 ihh   spread     pib_mensal 0.01     
2 ihh   selic      pib_mensal 0.01     
3 ihh   inad       pib_mensal 0.01     
4 ihh   pib_mensal prod_ind   0.01     
5 ihh   pib_mensal igp_di     0.01     
\end{verbatim}

\begin{verbatim}
# A tibble: 10 x 5
   V1    V2         V3         V4         resultado
   <chr> <chr>      <chr>      <chr>      <chr>    
 1 ihh   spread     selic      pib_mensal 0.01     
 2 ihh   spread     inad       pib_mensal 0.01     
 3 ihh   spread     pib_mensal prod_ind   0.01     
 4 ihh   spread     pib_mensal igp_di     0.01     
 5 ihh   selic      inad       pib_mensal 0.01     
 6 ihh   selic      pib_mensal prod_ind   0.01     
 7 ihh   selic      pib_mensal igp_di     0.01     
 8 ihh   inad       pib_mensal prod_ind   0.01     
 9 ihh   inad       pib_mensal igp_di     0.01     
10 ihh   pib_mensal prod_ind   igp_di     0.01     
\end{verbatim}

\begin{verbatim}
# A tibble: 11 x 6
   V1     V2     V3         V4         V5         resultado         
   <chr>  <chr>  <chr>      <chr>      <chr>      <chr>             
 1 ihh    spread selic      inad       pib_mensal 0.0101548982701054
 2 ihh    spread selic      pib_mensal prod_ind   0.01              
 3 ihh    spread selic      pib_mensal igp_di     0.01              
 4 ihh    spread inad       pib_mensal prod_ind   0.01              
 5 ihh    spread inad       pib_mensal igp_di     0.01              
 6 ihh    spread pib_mensal prod_ind   igp_di     0.01              
 7 ihh    selic  inad       pib_mensal prod_ind   0.01              
 8 ihh    selic  inad       pib_mensal igp_di     0.01              
 9 ihh    selic  pib_mensal prod_ind   igp_di     0.01              
10 ihh    inad   pib_mensal prod_ind   igp_di     0.01              
11 spread selic  inad       pib_mensal prod_ind   0.038977069480577 
\end{verbatim}

\begin{verbatim}
# A tibble: 5 x 7
  V1    V2     V3    V4         V5         V6       resultado         
  <chr> <chr>  <chr> <chr>      <chr>      <chr>    <chr>             
1 ihh   spread selic inad       pib_mensal prod_ind 0.0293105809705589
2 ihh   spread selic inad       pib_mensal igp_di   0.01              
3 ihh   spread selic pib_mensal prod_ind   igp_di   0.01              
4 ihh   spread inad  pib_mensal prod_ind   igp_di   0.01              
5 ihh   selic  inad  pib_mensal prod_ind   igp_di   0.01              
\end{verbatim}


\end{document}
