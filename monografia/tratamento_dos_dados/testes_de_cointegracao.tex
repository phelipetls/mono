\documentclass[]{article}
\usepackage{lmodern}
\usepackage{amssymb,amsmath}
\usepackage{ifxetex,ifluatex}
\usepackage{fixltx2e} % provides \textsubscript
\ifnum 0\ifxetex 1\fi\ifluatex 1\fi=0 % if pdftex
  \usepackage[T1]{fontenc}
  \usepackage[utf8]{inputenc}
\else % if luatex or xelatex
  \ifxetex
    \usepackage{mathspec}
  \else
    \usepackage{fontspec}
  \fi
  \defaultfontfeatures{Ligatures=TeX,Scale=MatchLowercase}
\fi
% use upquote if available, for straight quotes in verbatim environments
\IfFileExists{upquote.sty}{\usepackage{upquote}}{}
% use microtype if available
\IfFileExists{microtype.sty}{%
\usepackage{microtype}
\UseMicrotypeSet[protrusion]{basicmath} % disable protrusion for tt fonts
}{}
\usepackage[margin=1in]{geometry}
\usepackage{hyperref}
\hypersetup{unicode=true,
            pdftitle={Testes de cointegração},
            pdfborder={0 0 0},
            breaklinks=true}
\urlstyle{same}  % don't use monospace font for urls
\usepackage{graphicx,grffile}
\makeatletter
\def\maxwidth{\ifdim\Gin@nat@width>\linewidth\linewidth\else\Gin@nat@width\fi}
\def\maxheight{\ifdim\Gin@nat@height>\textheight\textheight\else\Gin@nat@height\fi}
\makeatother
% Scale images if necessary, so that they will not overflow the page
% margins by default, and it is still possible to overwrite the defaults
% using explicit options in \includegraphics[width, height, ...]{}
\setkeys{Gin}{width=\maxwidth,height=\maxheight,keepaspectratio}
\IfFileExists{parskip.sty}{%
\usepackage{parskip}
}{% else
\setlength{\parindent}{0pt}
\setlength{\parskip}{6pt plus 2pt minus 1pt}
}
\setlength{\emergencystretch}{3em}  % prevent overfull lines
\providecommand{\tightlist}{%
  \setlength{\itemsep}{0pt}\setlength{\parskip}{0pt}}
\setcounter{secnumdepth}{0}
% Redefines (sub)paragraphs to behave more like sections
\ifx\paragraph\undefined\else
\let\oldparagraph\paragraph
\renewcommand{\paragraph}[1]{\oldparagraph{#1}\mbox{}}
\fi
\ifx\subparagraph\undefined\else
\let\oldsubparagraph\subparagraph
\renewcommand{\subparagraph}[1]{\oldsubparagraph{#1}\mbox{}}
\fi

%%% Use protect on footnotes to avoid problems with footnotes in titles
\let\rmarkdownfootnote\footnote%
\def\footnote{\protect\rmarkdownfootnote}

%%% Change title format to be more compact
\usepackage{titling}

% Create subtitle command for use in maketitle
\providecommand{\subtitle}[1]{
  \posttitle{
    \begin{center}\large#1\end{center}
    }
}

\setlength{\droptitle}{-2em}

  \title{Testes de cointegração}
    \pretitle{\vspace{\droptitle}\centering\huge}
  \posttitle{\par}
    \author{}
    \preauthor{}\postauthor{}
    \date{}
    \predate{}\postdate{}
  

\begin{document}
\maketitle

\begin{center}\LARGE{Testes de co-integração para várias combinações das variáveis}\end{center}\begin{center}\large{Philips-Ouliaris}\end{center}

Quando o teste é feito somente com as variáveis diagnosticadas como não
estacionárias (isto é, retira-se igp e pib\_mensal, a inflação e a
atividade econômica), não se rejeita a hipótese nula de não-integração:

\[ \text{spread} = \beta_0 + \beta_1\text{selic} + \beta_2\text{inad} + \beta_3\text{pib\_mensal\*} + \beta_4\text{igp\_di\*} + \beta_5\text{ihh} \]

\begin{verbatim}
# A tibble: 6 x 3
  V1     V2        resultado
  <chr>  <chr>     <chr>    
1 ihh    spread    0.15     
2 ihh    selic     0.15     
3 ihh    inad_ipea 0.15     
4 spread selic     0.15     
5 spread inad_ipea 0.15     
6 selic  inad_ipea 0.15     
\end{verbatim}

\begin{verbatim}
# A tibble: 4 x 4
  V1     V2     V3        resultado
  <chr>  <chr>  <chr>     <chr>    
1 ihh    spread selic     0.15     
2 ihh    spread inad_ipea 0.15     
3 ihh    selic  inad_ipea 0.15     
4 spread selic  inad_ipea 0.15     
\end{verbatim}

\begin{verbatim}
# A tibble: 1 x 5
  V1    V2     V3    V4        resultado
  <chr> <chr>  <chr> <chr>     <chr>    
1 ihh   spread selic inad_ipea 0.15     
\end{verbatim}

~ ~ ~ ~

Porém, ao incluir as variáveis estacionárias, há evidência de que as
cinco co-integram:

~ ~ ~ ~

\begin{verbatim}
# A tibble: 1 x 7
  V1    V2     V3    V4         V5        V6     resultado
  <chr> <chr>  <chr> <chr>      <chr>     <chr>  <chr>    
1 ihh   spread selic pib_mensal inad_ipea igp_di 0.01     
\end{verbatim}

Pelo que eu entendi, o teste assume que todas as variáveis são I(1).
Logo, esse resultado não é válido.

\newpage

\begin{center}\LARGE{Testes de co-integração para várias combinações das variáveis}\end{center}\begin{center}\large{Engle-Granger}\end{center}

~ ~ ~

Pode-se chegar à mesma conclusão no teste de Engle-Granger. Mas aqui o
teste requer que se especifique qual é a variável dependente e as
variáveis indenpendentes:

~ ~ ~

\begin{verbatim}
Response: series %>% select(spread) %>% as.matrix 
Input: series %>% select(-spread, -date, -igp_di, -pib_mensal) %>% as.matrix 
Number of inputs: 3 
Model: y ~ X + 1 
------------------------------- 
Engle-Granger Cointegration Test 
alternative: cointegrated 

Type 1: no trend 
    lag      EG p.value 
   3.00   -2.01    0.10 
----- 
 Type 2: linear trend 
    lag      EG p.value 
  3.000   0.964   0.100 
----- 
 Type 3: quadratic trend 
    lag      EG p.value 
   3.00   -3.01    0.10 
----------- 
Note: p.value = 0.01 means p.value <= 0.01 
    : p.value = 0.10 means p.value >= 0.10 
\end{verbatim}


\end{document}
