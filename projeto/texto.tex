% ----
% Criação do documento com a classe abntex2
% ----
\documentclass[a4paper, article, 12pt, openany, oneside, english, brazil]{abntex2}

% ---
% Pacotes fundamentais
% ---
\usepackage{times}			% Usa a fonte Latin Modern
\usepackage[T1]{fontenc}		% Seleção de códigos de fonte.
\usepackage[utf8]{inputenc}		% Codificação do documento (conversão automática dos acentos)
\usepackage{indentfirst}		% Indenta o primeiro parágrafo de cada seção.
\usepackage{color}		        % Controle das cores
\usepackage{graphicx}			% Inclusão de gráficos/figuras.
\usepackage{subcaption}			% Inclusão de subfiguras.
\usepackage{microtype} 			% para melhorias de justificação
\usepackage{amsmath}			% Pacote matemático
\usepackage[brazil]{babel}

% ---

% ---
% Pacotes de citações
% ---
% \usepackage[brazilian,hyperpageref]{backref}	 % Paginas com as citações na bibl
\usepackage[alf]{abntex2cite}	                 % Citações padrão ABNT


% % ----
% % Configurações dos pacotes
% % ----
% % Configurações do pacote backref
% % Usado sem a opção hyperpageref de backref
% \renewcommand{\backrefpagesname}{Citado na(s) página(s):~}
% % Texto padrão antes do número das páginas
% \renewcommand{\backref}{}
% % Define os textos da citação
% \renewcommand*{\backrefalt}[4]{
% 	\ifcase #1 %
% 		Nenhuma citação no texto.%
% 	\or
% 		Citado na página #2.%
% 	\else
% 		Citado #1 vezes nas páginas #2.%
% 	\fi}%
% % ---


% ---
% Informações de dados para CAPA e FOLHA DE ROSTO
% ---
\autor{Phelipe Teles da Silva}
\titulo{Análise Quantitativa da Relação entre Spread e Concentração Bancária}
\data{2019}
\instituicao{Universidade Federal Rural do Rio de Janeiro}
\local{Seropédica, Rio de Janeiro}
\orientador[Orientadora:]{Débora Pimentel}
%\preambulo{}
%\tipotrabalho{}

\frenchspacing
% \setlength\afterchapskip{\lineskip}

\begin{document}
% ---
% Capa e Folha de Rosto
% ---
\imprimircapa
\imprimirfolhaderosto

% ---
% Insere o sumario automático
% ---
\pdfbookmark[0]{\contentsname}{toc}
\tableofcontents*
\cleardoublepage

\textual

\section{Problema}

    Quais são os principais determinantes do spread bancário no Brasil?

\section{Hipótese}
    


\section{Revisão de Literatura}
\subsection{Revisão da literatura teórica}

    Nesta seção pretendo explorar três modelos que se consagraram na literatura sobre a relação entre spread bancário e concentração bancária: o modelo do banco como uma firma maximizadora de lucros de \citeonline{klein}, o modelo do banco como intermediador financeiro de \citeonline{hoesaunders} e, finalmente, o modelo pós-keynesiano da preferência pela liquidez. O intuito é explicitar quais variáveis são relevantes para explicar o spread bancário e por quê, de um ponto de vista macro e microeconômico.

\subsubsection{O Banco como firma}

    Neste modelo, o banco é visto como uma firma que produz serviços voltados para intermediar a oferta e demanda de crédito, receber depósitos (D) e fazer empréstimos (L), a uma taxa de juros determinada em um mercado monopolístico ou semi-monopolístico, ou seja, em que tem poder de fixar a taxa de juros acima do custo marginal de produção, e é precisamente nesse sentido que o spread bancário é aqui entendido, como o poder de mercado deste banco. \cite{oreiro}

    O autor considera que o banco seja neutro ao risco, buscando maximizar tão somente o valor esperado do lucro, sem considerar sua variância, deparando-se com uma curva de demanda por empréstimos decrescente, $L(r_L)$, e uma curva de oferta de depósitos crescente, $D(r_D)$, e uma função custo do tipo $C(D, L)$, em que$ r_L$ denota a taxa de juros cobrada ao emprestar, e $r_D$ a paga ao depositante. A função custo, considerando as funções inversas, é assim considerada: \begin{equation}\pi(D, L) = r_L(L)L + rM - r_D(D)D - C(D, L)\end{equation} em que $r$ representa a taxa de juros do mercado interbancário, e M, o que o banco tem disponível para aplicar neste mercado, a saber, tudo o que recebe de depósito e não empresta nem vai para o compulsório (uma taxa $\alpha$), $M = (1 - \alpha)D - L$.

    E assim, manipulando a equação, obtemos o que representa o resultado da intermediação financeira subtraído de seus custos: \begin{equation}\pi(L, D) = (r_L(L) - r)L + (r(1 - \alpha) - r_D(D))D - C(D, L)\end{equation}

    O próximo passo é obter a margem ótima de intermediação, isto é, a que maximiza a função lucro. Para isso, tira-se as derivadas parciais desta função em relação a L e D para se chegar às condições de primeira ordem, e, depois de algumas manipulações algébricas, chega-se às equações fundamentais do modelo:

	\begin{equation}
	\frac{1}{\epsilon^{*}_L} = \frac{r^{*}_L - (r - C^{'}_L)}{r^{*}_L}
	\end{equation}

	\begin{equation}
	\frac{1}{\epsilon^{*}_D} = \frac{r(1-\alpha)-C^{'}_D - r^{*}_D}{r^{*}_D}
	\end{equation}

    O lado direito da equação é a versão para firma bancária do índice de Lerner\footnote{ Definido como a razão $(P - C_{mg}) / P$, em que $P$ é preço e $C_{mg}$ é o custo marginal, é uma medida do poder de mercado de um agente maximizador de lucros, sendo idêntico à recíproca da elasticidade. \cite{maudos}}, sendo, por definição, em cada um dos casos, igual ao inverso da elasticidade-juros da demanda por empréstimos ($\epsilon^{*}_L$) e oferta de depósitos ($\epsilon^{*}_D$). A interpretação desse resultado é que, para maximizar seus lucros, os bancos procuram fixar a taxa de juros num nível acima de seus custos, mas não ao ponto de perderem muitos clientes para a concorrência (o que acontece mais facilmente em mercados com demanda elástica etc.).

    Segue-se imediatamente destas equações que o spread bancário será tão maior quanto menos sensíveis forem as elasticidades da demanda por empréstimo e da oferta de depósito em relação à taxa de juros. Outra implicação interessante, e nada óbvia é de que, se a taxa de juros $r$ do mercado interbancário aumentar, as taxas de intermediação também irão \cite[p.~59]{freixas}.

\subsubsection{O banco como intermediador financeiro}

    Neste modelo, primeiro apresentado em \citeonline{hoesaunders}, o banco é considerado como um agente que atua como intermediador entre demandantes e ofertantes de fundos. É um dos mais influentes modelos na literatura, tendo sido extendido por vários outros autores, das quais se falará mais adiante de \citeonline{maudos}.

    Este modelo difere radicalmente do de \citeonline{klein} no que tange ao tipo de mercado em que o banco atua, não mais harmônico e equilibrado, mas sujeito a incertezas. Neste cenário, o banco não é mais considerado neutro ao risco, mas avesso a ele. Isso porque se depara com duas incertezas: o risco da inadimplência e o risco da taxa de juros, que tem a ver com a possível descoordenação entre demanda por empréstimos e ofertas de depósitos, caso em que o banco terá que recorrer ao mercado interbancário. No caso de demanda excessiva de empréstimos, o banco terá que pedir emprestado e estará sujeito ao risco da taxa de juros aumentar nesse ínterim. No caso de oferta excessiva de depósitos, terá que aplicar o excesso, no que estará sujeito ao risco da taxa cair. \citeonline[p.~2262]{maudos}

    Por esta razão, os bancos procuram minimizar esse risco fixando taxas de juros para os depósitos ($r_D$) e para os empréstimos ($r_L$) com a adição de uma pequena margem relativa à taxa de juros do mercado interbancário. Esta margem é o spread ($s$):

	\begin{gather}
		r_D = r - a \\
		r_L = r + b \\
		s = r_L - r_D = a + b
	\end{gather}

    Para uma derivação das equações do modelo, até a equação do spread ótimo, ver \citeonline[p.~2262]{maudos} ou \citeonline[p.~584]{hoesaunders}. Para propósitos da revisão, iremos omiti-la, considerando somente suas implicações no que concerne aos determinantes do spread. São eles:

	\begin{enumerate}
		\item Estrutura competitiva do mercado: o que se relaciona com as elasticidades-juros da demanda por empréstimo e da oferta de depósitos, como em \citeonline{klein}.
		\item Aversão ao risco: quanto mais avessos, maior o spread.
		\item Volatilidade da taxa de juros do mercado interbancário: quanto mais instável a taxa de juros, maior o risco, e portanto maior o spread.
		\item Inadimplência: quanto maior o risco de inadimplência, mais o bancos terão que se proteger via spread.
		\item Covariância entre risco da taxa de juros e de inadimplência: captura a influência da instabilidade macroeconômica na insolvência das famílias. \cite{oreiro}
		\item Tamanho médio das operações de crédito e de depósito: uma operação de tamanho maior que o médio significa maior perda potencial, da qual se protege com um spread maior.
	\end{enumerate}

    Esse modelo teórico é a base sob a qual se sustenta a abordagem empírica dos mesmos autores: a estimação do spread em dois estágios. Primeiro, estima-se o spread "puro", usando as variáveis não incluídas no modelo teórico como controle, comumente variáveis microeconômicas e características dos bancos, para então regredi-lo contra as variáveis do modelo teórico \citeonline{maudos}. Como será visto adiante, uma das vantanges dessa abordagem está em sua flexibilidade, permitindo a inclusão de variáveis sem grandes problemas \citeonline[p.~2]{almeida15}.

\subsection{Revisão da literatura empírica}

    Esta seção procura revisar os estudos empíricos sobre os determinantes do spread bancário que, cabe esclarecer, é calculado com base nas taxas de juros prefixadas cobradas nas modalidades de crédito com recursos livres e na taxa de captação de CDB de trinta dias.

    A estabilidade macroecônomica e as políticas governamentais que se seguiram ao Plano Real possibilitaram uma queda significativa do spread bancário, porém até um nível ainda bastante alto para os padrões internacionais. Somando-se a isso a reestruturação bancária observada no período, \citeonline{afanasieff02} notam que além da grande variação temporal do spread, há também uma grande variação interbancária. É nesse sentido em que os autores justificam o emprego de técnicas de dados em painel, com o objetivo de capturar essas características do mercado bancário brasileiro para decompor os principais determinantes do spread em variáveis micro e macroeconômicas.

    Para isso, os autores aplicam a metodologia de dois passos de \citeonline{hoesaunders}, usando dados mensais de 142 bancos para o período de fevereiro de 1997 a novembro de 2000, formando um painel com 5578 observações. Para a estimação do spread puro, usam como regressores o intercepto, um vetor temporal e um de características bancárias como controle \footnote{Dentre elas, número de agências, custos operacionais, a taxa de depósitos à vista e a prazo em relação ao ativo total, sendo duas variáveis distintas, a liquidez, pagamento de juros implícitos, alavancagem, uma variável dummy para banco estrangeiro, a receita de serviços sobre a receita operacional total e o patrimônio líquido}. Uma vez estimado o spread puro, dado nesse caso pelos coeficientes do vetor temporal e do intercepto, ele é então regredido contra as variáveis macroeconômicas: a taxa de inflação, o compulsório, a taxa SELIC, o crescimento do PIB, a taxação financeira e uma proxy para prêmio de risco.

    No primeiro estágio, foi encontrado que o custo operacional, a taxa de depósitos à vista em relação ao ativo total e a de receita de serviços em relação à receita operacional afetam positivamente o spread, ao passo que os bancos estrangeiros estão em média associados a um spread menor. Os autores, ao contraporem o spread puro estimado com o observado, apontam ainda que os fatores microeconômicos não parecem ser os principais determinantes do spread.
    
    Pelos resultados do segundo estágio, há uma relação direta e significativa entre spread e taxa básica de juros, risco, taxação financeira e crescimento do PIB, e contrária no caso da inflação. O efeito estimado do compulsório não foi significativo. Os autores, por fim, concluem que este é um resultado coerente com a estabilização macroeconômica da época, mais chamam a atenção para a ineficiência e disparidade no mercado bancário, que tornam possível que bancos atuem cobrando taxas muito maiores que seus rivais.

    Seguindo linha metodológica parecida, \citeonline{bignotto06} também se utilizam do modelo teórico de \citeonline{hoesaunders} para analisar a influência dos custos de intermediação e dos fatores de risco de crédito e de juros sobre o spread. Para a estimação, fez-se uso de dados de 87 bancos brasileiros ao longo do 1º trimestre de 2001 até o 1º trimestre de 2004, formando uma base de dados em painel desbalanceado com 1131 observações.
    
    Foram usados como regressores o custo administrativo, uma proxy para o risco de crédito e de juros, o market-share, o risco de liquidez, a taxa de receitas de serviços sobre o ativo total e a da despesa tributária sobre o ativo total, o compulsório, a SELIC, a inflação e o ativo total de cada banco corrigido pela inflação, além de uma variável não-observável de aversão ao risco como controle. 
    
    O modelo que os autores julgam mais eficiente revela uma relação direta entre o spread e os custos administrativos, o risco de crédito e de juros, o compulsório, a SELIC, e o ativo total, todos com o sinal esperado. Algumas variáveis de controle apresentaram coeficientes inesperados, como no caso da liquidez, da receita de serviços e do market share. Já a variável relativa à carga tributária não se mostrou significativa.

    Complementando os importantes estudos feitos pelo Banco Central do Brasil à época, que consistiam na decomposição contábil do spread, \citeonline{nakane02} buscaram investigar a sensibilidade do spread a variações nas variáveis que o compõem, isto é, investigar o assunto por um abordagem econométrica. Para isso, os autores utilizaram a seguinte equação:

    $$\ln spread = \beta_0Tend + \beta_1\ln selic_t + \beta_2\ln adm_t + \beta_3\ln risk_t + \beta_4\ln imp_t + \beta_4\ln comp_t$$

    Em que $selic$ é a taxa básica de juros, $adm$ é a razão das despesas adminsitrativas sobre o volume de crédito, $risk$ é uma proxy de risco global\footnote{"[É ] o spread do rendimento do C-Bond sobre o rendimento do título do Tesouro americano com mesma maturidade." \cite[p.~10]{nakane02}} que busca capturar expectativas, algo mais apropriado quando se trata do spread ex-ante, $imp$ é uma medida da incidência de tributos indiretos sobre o spread, $comp$ é a taxa de compulsório sobre depósitos à vista exigida pelo Bacen,  $Tend$ é uma medida tendência determinista incluída com o intuito de controlar para variáveis que não foram incluídas na equação, como a taxa de inflação, o nível de atividade econômica etc. e $\ln$ é o logaritmo natural.  Para mais detalhes, ver \citeonline[p.~10]{nakane02}

    Para a estimação da equação de longo prazo, foram usados dados mensais do período de agosto de 1994 a setembro de 2001 em um modelo de vetores autorregressivos (VAR) que, uma vez tratado estatisticamente, revelou que a taxa Selic, as despesas adminsitrativas, o risco e os impostos indiretos afetam positiva e significativamente o spread. Já o compulsório não se mostrou significativo e foi retirado do modelo.

    Em seguida, os autores decompuseram o spread do período com base nessa estimação. Para isso, isolaram o efeito da variável de tendência determinista e suporam que o spread pode ser bem aproximado pela equação de longo prazo. O resultado indica que todas as variáveis são bem relevantes na composição do spread no período, em especial a variável de risco, relevância esta que foi aumentando com o passar dos anos e que é análoga à considerável participação da inadimplência na decomposição contábil do spread. Os custos administrativos e a Selic vêm em seguida em importância, ambas com tendências sutilmente declinantes. E, por fim, a importância relativa dos impostos indiretos é a menos sobressalente, mas aumentou com os anos. 

    Um artigo que segue um caminho semelhante em termos de modelagem é o de \citeonline{oreiro}, já que para a estimação dos determinantes também se utilizou um modelo VAR. Este estudo, porém, se distingue por dar foco aos efeitos das variáveis macroeconômicas sobre o spread para o período de janeiro de 1995 a dezembro de 2003. 
    
    São usados como regressores séries de frequência mensal como o nível da taxa básica de juros (SELIC) e sua volatilidade (como proxy para o risco de juros), o nível de atividade econômica, o recolhimento compulsório sobre depósitos à vista e a inflação.

    Para interpretar os coeficientes estimados, os autores se utilizaram da função de impulso-resposta, em que se procurou observar o efeito de longo prazo que um choque exógeno de um desvio-padrão em uma variável independente tem sobre a variável dependente. A decomposição da variância também foi utilizada, com objetivo similar. Em ambos os casos, analisou-se o efeito do choque ao longo de 12 meses.

    Os resultados mostram que um choque exógeno no nível e volatilidade da taxa de juros tem um efeito positivo e persistente sobre o spread, como esperado. Um choque na proxy para o nível de atividade econômica mostrou ter um impacto positivo, o que os autores interpretaram como um possível efeito do poder de mercado dos bancos prevalecendo sobre o efeito inadimplência. O efeito de um choque na inflação se mostrou insignificante estatisticamente. Eles concluem dizendo que um ambiente macroeconômico estável é uma condição indispensável para reduzir o spread.



\bibliography{bibliografia}

\end{document}
