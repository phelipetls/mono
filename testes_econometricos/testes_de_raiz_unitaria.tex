\documentclass[]{article}
\usepackage{lmodern}
\usepackage{amssymb,amsmath}
\usepackage{ifxetex,ifluatex}
\usepackage{fixltx2e} % provides \textsubscript
\ifnum 0\ifxetex 1\fi\ifluatex 1\fi=0 % if pdftex
  \usepackage[T1]{fontenc}
  \usepackage[utf8]{inputenc}
\else % if luatex or xelatex
  \ifxetex
    \usepackage{mathspec}
  \else
    \usepackage{fontspec}
  \fi
  \defaultfontfeatures{Ligatures=TeX,Scale=MatchLowercase}
\fi
% use upquote if available, for straight quotes in verbatim environments
\IfFileExists{upquote.sty}{\usepackage{upquote}}{}
% use microtype if available
\IfFileExists{microtype.sty}{%
\usepackage{microtype}
\UseMicrotypeSet[protrusion]{basicmath} % disable protrusion for tt fonts
}{}
\usepackage[margin=1in]{geometry}
\usepackage{hyperref}
\hypersetup{unicode=true,
            pdftitle={Testes de raiz unitária},
            pdfborder={0 0 0},
            breaklinks=true}
\urlstyle{same}  % don't use monospace font for urls
\usepackage{longtable,booktabs}
\usepackage{graphicx,grffile}
\makeatletter
\def\maxwidth{\ifdim\Gin@nat@width>\linewidth\linewidth\else\Gin@nat@width\fi}
\def\maxheight{\ifdim\Gin@nat@height>\textheight\textheight\else\Gin@nat@height\fi}
\makeatother
% Scale images if necessary, so that they will not overflow the page
% margins by default, and it is still possible to overwrite the defaults
% using explicit options in \includegraphics[width, height, ...]{}
\setkeys{Gin}{width=\maxwidth,height=\maxheight,keepaspectratio}
\IfFileExists{parskip.sty}{%
\usepackage{parskip}
}{% else
\setlength{\parindent}{0pt}
\setlength{\parskip}{6pt plus 2pt minus 1pt}
}
\setlength{\emergencystretch}{3em}  % prevent overfull lines
\providecommand{\tightlist}{%
  \setlength{\itemsep}{0pt}\setlength{\parskip}{0pt}}
\setcounter{secnumdepth}{0}
% Redefines (sub)paragraphs to behave more like sections
\ifx\paragraph\undefined\else
\let\oldparagraph\paragraph
\renewcommand{\paragraph}[1]{\oldparagraph{#1}\mbox{}}
\fi
\ifx\subparagraph\undefined\else
\let\oldsubparagraph\subparagraph
\renewcommand{\subparagraph}[1]{\oldsubparagraph{#1}\mbox{}}
\fi

%%% Use protect on footnotes to avoid problems with footnotes in titles
\let\rmarkdownfootnote\footnote%
\def\footnote{\protect\rmarkdownfootnote}

%%% Change title format to be more compact
\usepackage{titling}

% Create subtitle command for use in maketitle
\providecommand{\subtitle}[1]{
  \posttitle{
    \begin{center}\large#1\end{center}
    }
}

\setlength{\droptitle}{-2em}

  \title{Testes de raiz unitária}
    \pretitle{\vspace{\droptitle}\centering\huge}
  \posttitle{\par}
    \author{}
    \preauthor{}\postauthor{}
    \date{}
    \predate{}\postdate{}
  

\begin{document}
\maketitle

\begin{center} \large{Gráficos de todas as séries} \end{center}

Cada série conta com 82 observações, de periodicidade mensal.

\includegraphics{testes_de_raiz_unitaria_files/figure-latex/plot_de_todas_as_series-1.pdf}

\begin{center} \large{Séries diagnosticadas como estacionárias em cada teste} \end{center}

\begin{longtable}[]{@{}llll@{}}
\toprule
variavel & teste & pvalor & diagnostico\tabularnewline
\midrule
\endhead
igp\_di & adf & 0.05 & estacionário\tabularnewline
pib\_mensal & pp & 0.01 & estacionário\tabularnewline
igp\_di & pp & 0.01 & estacionário\tabularnewline
igp\_di & kpss & 0.10 & estacionário\tabularnewline
\bottomrule
\end{longtable}

\begin{longtable}[]{@{}llllll@{}}
\toprule
variavel & criterio & tipo & estatistica\_teste & valor\_critico &
diagnostico\tabularnewline
\midrule
\endhead
igp\_di & AIC & none & -3.16 & -1.95 & estacionário\tabularnewline
igp\_di & AIC & drift & -4.50 & -2.89 & estacionário\tabularnewline
pib\_mensal & AIC & trend & -4.23 & -3.45 & estacionário\tabularnewline
igp\_di & AIC & trend & -4.49 & -3.45 & estacionário\tabularnewline
\bottomrule
\end{longtable}


\end{document}
